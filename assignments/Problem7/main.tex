\documentclass[journal]{IEEEtran}
\usepackage[a5paper, margin=10mm, onecolumn]{geometry}
%\usepackage{lmodern} % Ensure lmodern is loaded for pdflatex
\usepackage{tfrupee} % Include tfrupee package

\setlength{\headheight}{1cm} % Set the height of the header box
\setlength{\headsep}{0mm}     % Set the distance between the header box and the top of the text

\usepackage{gvv-book}
\usepackage{gvv}
\usepackage{cite}
\usepackage{amsmath,amssymb,amsfonts,amsthm}
\usepackage{algorithmic}
\usepackage{graphicx}
\usepackage{textcomp}
\usepackage{xcolor}
\usepackage{txfonts}
\usepackage{listings}
\usepackage{enumitem}
\usepackage{mathtools}
\usepackage{gensymb}
\usepackage{comment}
\usepackage[breaklinks=true]{hyperref}
\usepackage{tkz-euclide} 
\usepackage{listings}
% \usepackage{gvv}                                        
\def\inputGnumericTable{}                                 
\usepackage[latin1]{inputenc}                                
\usepackage{color}                                            
\usepackage{array}                                            
\usepackage{longtable}                                       
\usepackage{calc}                                             
\usepackage{multirow}                                         
\usepackage{hhline}                                           
\usepackage{ifthen}                                           
\usepackage{lscape}
\begin{document}

\bibliographystyle{IEEEtran}
\vspace{3cm}

\title{10.4.1.1.8}
\author{EE24BTECH11024 - G. Abhimanyu Koushik}
 \maketitle
% \newpage
% \bigskip
{\let\newpage\relax\maketitle}

\renewcommand{\thefigure}{\theenumi}
\renewcommand{\thetable}{\theenumi}
\setlength{\intextsep}{10pt} % Space between text and floats


\numberwithin{equation}{enumi}
\numberwithin{figure}{enumi}
\renewcommand{\thetable}{\theenumi}


\textbf{Question}:\\
Find the roots of the equation $x^3 - 4x^2 - x + 1 = \brak{x - 2}^3$
\\
\textbf{Solution: }\\
Theoritical solution:\\
The equation can be simplified to
\begin{align}
	x^3 - 4x^2 - x + 1 &= \brak{x - 2}^3\\
	x^3 - 4x^2 - x + 1 &= x^3-6x^2+12x-8\\
	2x^2 - 13x + 9 &= 0
\end{align}
Applying quadratic formula gives solution as
\begin{align}
	x_1 = \frac{13-\sqrt{97}}{4}\\
	x_2 = \frac{13+\sqrt{97}}{4}
\end{align}
Computational solution:\\
Two methods to find solution of a quadratic equation are:\\
Matrix-Based Method:\\
For a polynomial equation of form $x_n+b_{n-1}x^{n-1}+\dots+b_2x^2+b_1x+b_0 = 0$ we construct a matrix called companion matrix of form
\begin{align}
	\Lambda = \myvec{0&1&0&\dots&0\\ 0&0&1&\dots&0\\ \vdots &\vdots &\vdots &\ddots&\vdots\\0&0&0&\vdots&1\\-b_0&-b_1&-b_2&\dots&-b_{n-1}}
\end{align}
The eigenvalues of this matrix are the roots of the given polynomial equation.\\
The solution given by the code is
\begin{align}
	x_1 = 0.7878\\
	x_2 = 5.7122
\end{align}
Newton-Raphson Method:\\
Start with an initial guess $x_0$, and then run the following logical loop,
\begin{align}
    x_{n+1} = x_n - \frac{f\brak{x_n}}{f^{\prime}\brak{x_n}} 
\end{align}
where,
\begin{align}
    f\brak{x} = 2x^2 - 13x + 9\\
    f^{\prime}\brak{x} = 4x-13
\end{align}
The problem with this method is if the roots are complex but the coeffcients are real, $x_n$ either converges to an extrema or grows continuously without any bound.
To get the complex solutions, however , we can just take the initial guess point to be a 
random complex number.\\
The output of a program written to find roots is shown below:
\begin{align}
	r_1 = 0.7878\\
	r_2 = 5.7122
\end{align}
\end{document}
