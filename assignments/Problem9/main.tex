\documentclass[journal]{IEEEtran}
\usepackage[a5paper, margin=10mm, onecolumn]{geometry}
%\usepackage{lmodern} % Ensure lmodern is loaded for pdflatex
\usepackage{tfrupee} % Include tfrupee package

\setlength{\headheight}{1cm} % Set the height of the header box
\setlength{\headsep}{0mm}     % Set the distance between the header box and the top of the text

\usepackage{gvv-book}
\usepackage{gvv}
\usepackage{cite}
\usepackage{amsmath,amssymb,amsfonts,amsthm}
\usepackage{algorithmic}
\usepackage{graphicx}
\usepackage{textcomp}
\usepackage{xcolor}
\usepackage{txfonts}
\usepackage{listings}
\usepackage{enumitem}
\usepackage{mathtools}
\usepackage{gensymb}
\usepackage{comment}
\usepackage[breaklinks=true]{hyperref}
\usepackage{tkz-euclide} 
\usepackage{listings}
% \usepackage{gvv}                                        
\def\inputGnumericTable{}                                 
\usepackage[latin1]{inputenc}                                
\usepackage{color}                                            
\usepackage{array}                                            
\usepackage{longtable}                                       
\usepackage{calc}                                             
\usepackage{multirow}                                         
\usepackage{hhline}                                           
\usepackage{ifthen}                                           
\usepackage{lscape}
\begin{document}

\bibliographystyle{IEEEtran}
\vspace{3cm}

\title{10.3.2.4.4}
\author{EE24BTECH11024 - G. Abhimanyu Koushik}
 \maketitle
% \newpage
% \bigskip
{\let\newpage\relax\maketitle}

\renewcommand{\thefigure}{\theenumi}
\renewcommand{\thetable}{\theenumi}
\setlength{\intextsep}{10pt} % Space between text and floats


\numberwithin{equation}{enumi}
\numberwithin{figure}{enumi}
\renewcommand{\thetable}{\theenumi}


\textbf{Question}:\\
A three coins are tossed once, what is the probability of getting atmost 2 heads?
\\
\textbf{Solution: }\\
Define a discrete random variable X = number of heads\newline
We will assume our random variable as a sum of outcomes of three bernoulli random variables
\begin{align}
	X = X_1+X_2+X_3
\end{align}
Where
\begin{align}
X_i = 
\begin{cases}
	1, & \text{Outcome in Heads}\\
	0, & \text{Outcome in Tails}
\end{cases}\\
p_{X_i}(n) = 
\begin{cases}
	1-p, & n = 0\\
	p, & n = 1
\end{cases}
\end{align}
Where $p=\frac{1}{2}$\newline
Using properties of Z-Transform of PMF
\begin{align}
	M_X(z) &= M_{X_1}(z)M_{X_2}(z)M_{X_3}(z)\\
	M_{X_1}(z) &= \sum_{n=-\infty}^{\infty}p_{X_1}(n)z^{-n} = p+(1-p)z^{-1}\\
	M_{X_2}(z) &= \sum_{n=-\infty}^{\infty}p_{X_2}(n)z^{-n} = p+(1-p)z^{-1}\\
	M_{X_3}(z) &= \sum_{n=-\infty}^{\infty}p_{X_3}(n)z^{-n} = p+(1-p)z^{-1}\\
	M_X(z) &= (p+(1-p)z^{-1})^3\\
	 &= \sum_{k=-\infty}^{\infty}\comb{3}{k}p^{3-k}(1-p)^kz^{-k}\\
	p_{X}(k) &= \comb{3}{k}p^{3-k}(1-p)^{k}\\
	p_{X}(k) &= \frac{\comb{3}{k}}{8}
\end{align}
The Probability Mass Function (PMF) for the given random variable is
\begin{align}
p_X(n) =
\begin{cases}
	\frac{1}{8}, & n = 0 \\
	\frac{3}{8}, & n = 1 \\
	\frac{3}{8}, & n = 2 \\
	\frac{1}{8}, & n = 3 \\
\end{cases}
\end{align}
The Cumulative Distribution Function (CDF) for the given random variable is
\begin{align}
F_X(n) = p(X \le n) = 
\begin{cases}
	0, & n < 0 \\
	\frac{1}{8}, & 0 \le n < 1 \\
	\frac{4}{8}, & 1 \le n < 2 \\
	\frac{7}{8}, & 2 \le n < 3 \\
	1, & 3 \le n\\
\end{cases}
\end{align}
The probability of getting atmost 2 heads is
\begin{align}
  F_X\brak{2} &= \frac{7}{8}
\end{align}
Simulation:
\newline
To run a simulation we need to generate random numbers with uniform probability, which is done
as shown below(Algorithm taken from OpenSSL's random\_uniform.c):
\begin{enumerate}
  \item \text{Generate 32 bits of entropy using /dev/urandom.}
  \item Treat this as a fixed point number in the range [0, 1)
  \item Scale this to desired range using fixed point multiplication and treat as 64bit number(upper 32 bits integer and rest as fractional part)
  \item Return the integer part of the fixed point numbers
\end{enumerate}
The following shows how the relative frequency reaches true probability with increasing number of trials of the event.
\begin{figure}[h!]
   \centering
   \includegraphics[width=1\columnwidth]{figs/fig.png}
    \caption{Relative Frequency tends to True Probability}
\end{figure}
\begin{figure}[h!]
   \centering
   \includegraphics[width=1\columnwidth]{figs/pmf.png}
    \caption{Probability Mass Function of given Random variable}
\end{figure}
\begin{figure}[h!]
   \centering
   \includegraphics[width=1\columnwidth]{figs/cdf.png}
    \caption{Cumulative Distribution Function of given Random variable}
\end{figure}
\end{document}
