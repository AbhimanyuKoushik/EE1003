\documentclass{beamer}
\mode<presentation>
\usepackage{amsmath}
\usepackage{amssymb}
%\usepackage{advdate}
\usepackage{adjustbox}
\usepackage{subcaption}
\usepackage{enumitem}
\usepackage{multicol}
\usepackage{gensymb}
\usepackage{mathtools}
\usepackage{listings}
\usepackage{url}
\def\UrlBreaks{\do\/\do-}
\usetheme{Boadilla}
\usecolortheme{lily}
\setbeamertemplate{footline}
{
  \leavevmode%
  \hbox{%
  \begin{beamercolorbox}[wd=\paperwidth,ht=2ex,dp=1ex,right]{author in head/foot}%
    \insertframenumber{} / \inserttotalframenumber\hspace*{2ex} 
  \end{beamercolorbox}}%
  \vskip0pt%
}
\setbeamertemplate{navigation symbols}{}

\providecommand{\nCr}[2]{\,^{#1}C_{#2}} % nCr
\providecommand{\nPr}[2]{\,^{#1}P_{#2}} % nPr
\providecommand{\mbf}{\mathbf}
\providecommand{\pr}[1]{\ensuremath{\Pr\left(#1\right)}}
\providecommand{\qfunc}[1]{\ensuremath{Q\left(#1\right)}}
\providecommand{\sbrak}[1]{\ensuremath{{}\left[#1\right]}}
\providecommand{\lsbrak}[1]{\ensuremath{{}\left[#1\right.}}
\providecommand{\rsbrak}[1]{\ensuremath{{}\left.#1\right]}}
\providecommand{\brak}[1]{\ensuremath{\left(#1\right)}}
\providecommand{\lbrak}[1]{\ensuremath{\left(#1\right.}}
\providecommand{\rbrak}[1]{\ensuremath{\left.#1\right)}}
\providecommand{\cbrak}[1]{\ensuremath{\left\{#1\right\}}}
\providecommand{\lcbrak}[1]{\ensuremath{\left\{#1\right.}}
\providecommand{\rcbrak}[1]{\ensuremath{\left.#1\right\}}}
\theoremstyle{remark}
\newtheorem{rem}{Remark}
\newcommand{\sgn}{\mathop{\mathrm{sgn}}}
\providecommand{\abs}[1]{\left\vert#1\right\vert}
\providecommand{\res}[1]{\Res\displaylimits_{#1}} 
\providecommand{\norm}[1]{\lVert#1\rVert}
\providecommand{\mtx}[1]{\mathbf{#1}}
\providecommand{\mean}[1]{E\left[ #1 \right]}
\providecommand{\fourier}{\overset{\mathcal{F}}{ \rightleftharpoons}}
%\providecommand{\hilbert}{\overset{\mathcal{H}}{ \rightleftharpoons}}
\providecommand{\system}{\overset{\mathcal{H}}{ \longleftrightarrow}}
	%\newcommand{\solution}[2]{\textbf{Solution:}{#1}}
%\newcommand{\solution}{\noindent \textbf{Solution: }}
\providecommand{\dec}[2]{\ensuremath{\overset{#1}{\underset{#2}{\gtrless}}}}
\newcommand{\myvec}[1]{\ensuremath{\begin{pmatrix}#1\end{pmatrix}}}
\let\vec\mathbf

\lstset{
%language=C,
frame=single, 
breaklines=true,
columns=fullflexible
}

\numberwithin{equation}{section}

\title{12.9.3.11.3 Presentation}
\author{G. Abhimanyu Koushik \\ EE24BTECH11024}

\date{\today} 
\begin{document}

\begin{frame}
\titlepage
\end{frame}

\section*{Outline}
\begin{frame}
\tableofcontents
\end{frame}
\section{Problem}
\begin{frame}
\frametitle{Problem Statement}
%
Solve the differential equation $\frac{d^2y}{dx^2} + 1 = 0$ with initial conditions $y\brak{0} = 0$ and $y^{\prime}\brak{0} = 0$
%
\end{frame}

%\subsection{Literature}
\section{Solution}
\subsection{Input Parameters}
\begin{frame}
\frametitle{Input Parameters}
%\framesubtitle{Literature}
\begin{table}[H]    
  \centering
  \begin{tabular}[12pt]{ |c| c|}
    \hline
    \textbf{Variable} & \textbf{Description}\\ 
    \hline
    $c_1$ &First Integration constant\\
    \hline
    $c_2$ &Second Integration constant\\
    \hline
    $n$ & Order of given differential equation\\
    \hline
    $a_i$ & Coeefficient of $i$th derivative of the function in the equation\\
    \hline
    $c$ & constant in the equation\\
    \hline
    \end{tabular}

\end{table}
\end{frame}
\subsection{Laplace Transform properties}
\begin{frame}
\frametitle{Laplace Transform properties}
%\framesubtitle{Literature}
Properties of Laplace tranform
\begin{align}
	\mathcal{L}\brak{y^{\prime\prime}} &= s^2\mathcal{L}\brak{y} -sy\brak{0}-y^\prime\brak{0}\\
	\mathcal{L}\brak{1} &= \frac{1}{s}\\
	\mathcal{L}^{-1}\brak{\frac{2}{s^3}} &= x^2u\brak{x}\\
	\mathcal{L}\brak{cf\brak{t}} &= c\mathcal{L}\brak{f\brak{t}}
\end{align}
%
\end{frame}
\subsection{Equation solving}
\begin{frame}
\frametitle{Equation solving}
Applying the properties to the given equation
\begin{align}
	y^{\prime\prime} + 1 &= 0\\
	\mathcal{L}\brak{y^{\prime\prime}} + \mathcal{L}\brak{1} &= 0\\
	s^2\mathcal{L}\brak{y} -sy\brak{0}-y^\prime\brak{0}+\frac{1}{s} &= 0\\
\end{align}
Substituting the initial conditions gives
\begin{align}
	s^3\mathcal{L}\brak{y} + 1 &= 0\\
	\mathcal{L}\brak{y} &= \frac{-1}{s^3}\\
	y &= \frac{-1}{2}\mathcal{L}^{-1}\brak{\frac{2}{s^3}}\\
	y &= \frac{-1}{2}x^2u\brak{x}
\end{align}
\end{frame}
\begin{frame}
The theoritical solution is 
\begin{align}
	f\brak{x} = \frac{-x^2}{2}u\brak{x}
\end{align}
\end{frame}


\subsection{Computational Solution}
\begin{frame}
\frametitle{Computational Solution}
Consider the given linear differential equation
\begin{align}
	a_{n}y^n + a_{n-1}y^{n-1} + \dots + a_{1}y^\prime + a_{0}y + c = 0
\end{align}
Then
\begin{align}
	y^{\prime}\brak{t} = \lim_{h\to 0}\frac{y\brak{t+h} - y\brak{t}}{h}\\
	y\brak{t+h} = y\brak{t} + hy^{\prime}\brak{t}
\end{align}
Similarly
\begin{align}
	y^{i}\brak{t+h} &= y^{i}\brak{t} + hy^{i+1}\brak{t}\\
	y^{n-1}\brak{t+h} &= y^{n-1}\brak{t} + hy^{n}\brak{t}\\
	y^{n-1}\brak{t+h} &= y^{n-1}\brak{t} + h\brak{-\frac{a_{n-1}}{a_n}y^{n-1}-\frac{a_{n-2}}{a_n}y^{n-2} - \dots -\frac{a_{0}}{a_n}y - \frac{c}{a_n}}
\end{align}
\end{frame}
\begin{frame}
Where i ranges from 0 to $n-1$\\
\begin{align}
	\vec{y}\brak{t+h} = \vec{y}\brak{t} + \myvec{0 & 0 & 0 & 0 & \dots & 0 & 0\\ 0 & 0 & 1 & 0 & \dots & 0 & 0\\0 & 0 & 0 & 1 & \dots & 0 & 0\\\vdots & \vdots & \vdots & \vdots& \ddots & \vdots & \vdots\\
	0 & 0 & 0 & 0 & \dots & 0 & 1\\-\frac{1}{a_n} & -\frac{a_0}{a_n} & -\frac{a_1}{a_n} & -\frac{a_2}{a_n} & \dots & -\frac{a_{n-2}}{a_n} & -\frac{a_{n-1}}{a_n}}\brak{h\vec{y}\brak{t}}\\
	\vec{y}\brak{t+h} = \myvec{1 & 0 & 0 & 0 & \dots & 0 & 0\\ 0 & 1 & h & 0 & \dots & 0 & 0\\0 & 0 & 1 & h & \dots & 0 & 0\\\vdots & \vdots & \vdots & \vdots& \ddots & \vdots & \vdots\\
	0 & 0 & 0 & 0 & \dots & 1 & h\\-\frac{h}{a_n} & -\frac{a_0h}{a_n} & -\frac{a_1h}{a_n} & -\frac{a_2h}{a_n} & \dots & -\frac{a_{n-2}h}{a_n} & 1-\frac{a_{n-1}h}{a_n}}\brak{\vec{y}\brak{t}}
\end{align}
\end{frame}
\begin{frame}
Discretizing the steps gives us
\begin{align}
	\vec{y}_{k+1} = \myvec{1 & 0 & 0 & 0 & \dots & 0 & 0\\ 0 & 1 & h & 0 & \dots & 0 & 0\\0 & 0 & 1 & h & \dots & 0 & 0\\\vdots & \vdots & \vdots & \vdots& \ddots & \vdots & \vdots\\
	0 & 0 & 0 & 0 & \dots & 1 & h\\-\frac{h}{a_n} & -\frac{a_0h}{a_n} & -\frac{a_1h}{a_n} & -\frac{a_2h}{a_n} & \dots & -\frac{a_{n-2}h}{a_n} & 1-\frac{a_{n-1}h}{a_n}}\brak{\vec{y}_{k}}
\end{align}
where $k$ ranges from 0 to number of data points with $y^{i}_0$ being the given initial condition and vector $\vec{y}_0 = \myvec{c\\y\brak{0}\\y^\prime\brak{0}\\\vdots\\y^{n-1}\brak{0}}$
\end{frame}
\begin{frame}
For the given question\\
\begin{align}
	\vec{y}_{k+1} = \myvec{1 & 0 & 0\\ 0 & 1 & h\\ -h & 0 & 1}\vec{y}_k
\end{align}
Record the $y_k$ for 
\begin{align}
x_k =lowerbound+kh
\end{align}
and then plot the graph. The result will be as given below.\\
The codes below verifies the obtained solution. 
\end{frame}
%\section{Plot}
\section{Plot of the function}
\begin{frame}
\frametitle{Plot of the function}
\begin{figure}[H]
    \centering
	\includegraphics[width=0.8\columnwidth]{figs/fig.png}
    \caption{Function satisfying given differential equation}
    \end{figure}   
\end{frame}
\section{C Code}
\begin{frame}[fragile]
\frametitle{C Code}
\begin{lstlisting}[language=C]
#include <stdio.h>
#include <stdlib.h>
#include <math.h>
#include "functions.h"
double** matrixgen(int order, double coefficients[order+2], double stepsize){
	double** outputmatrix = identity(order+1);
	for(int i=1; i<order; i++){
		outputmatrix[i][i+1] = stepsize;
	}
	outputmatrix[order][0] = -1/coefficients[0]*stepsize;
	for(int i=1; i<order+1; i++){
		outputmatrix[order][i] = (-coefficients[order+1-i]/coefficients[0])*stepsize;
	}
	outputmatrix[order][order] += 1; 
	return outputmatrix;}

\end{lstlisting}
\end{frame}
\begin{frame}[fragile]
\begin{lstlisting}[language=C]

double* recorddata(double lowerbound, double upperbound, int order,  double coefficients[order+2], double initialconditions[order], double stepsize){
	double** vector_y = createMat(order+1,1);
	vector_y[0][0] = coefficients[order+1];
	for(int i=0;i<order;i++){
		vector_y[i+1][0] = initialconditions[i];
	}
	double** matrix = matrixgen(order, coefficients, stepsize);
	int no_datapoints = ((upperbound-lowerbound)/stepsize);
	double* yvalues = malloc(no_datapoints*sizeof(double));
	for(int i = 0; i<no_datapoints; i++){
		vector_y = Matmul(matrix,vector_y,order+1,order+1,1);
		yvalues[i] = vector_y[1][0];
	}
	return yvalues;
}


\end{lstlisting}
\end{frame}

\section{Python Code}
\begin{frame}[fragile]
\frametitle{Python Code for Plotting}
\begin{lstlisting}[language=Python]
import ctypes
import numpy as np
import matplotlib.pyplot as plt

# Load the shared library
solver = ctypes.CDLL('./solver.so')

# Define the function signatures
solver.recorddata.restype = ctypes.POINTER(ctypes.c_double)
solver.recorddata.argtypes = [
    ctypes.c_double,  # lowerbound
    ctypes.c_double,  # upperbound
    ctypes.c_int,     # order
    ctypes.POINTER(ctypes.c_double),  # coefficients
    ctypes.POINTER(ctypes.c_double),  # initialconditions
    ctypes.c_double   # stepsize
]

\end{lstlisting}
\end{frame}
\begin{frame}[fragile]
\begin{lstlisting}[language=Python]

# Define parameters
order = 2
lowerbound = 0.0
upperbound = 10.0
stepsize = 0.001
coefficients = np.array([1.0, 0.0, 0.0, 1.0], dtype=np.double) 
initialconditions = np.array([0.0, 0.0], dtype=np.double)

# Calculate the number of data points
no_datapoints = int((upperbound - lowerbound) / stepsize)

# Call the C function
results_ptr = solver.recorddata(
    ctypes.c_double(lowerbound),
    ctypes.c_double(upperbound),
    ctypes.c_int(order),
    coefficients.ctypes.data_as(ctypes.POINTER(ctypes.c_double)),
    initialconditions.ctypes.data_as(ctypes.POINTER(ctypes.c_double)),

\end{lstlisting}
\end{frame}
\begin{frame}[fragile]
\begin{lstlisting}[language=Python]
    ctypes.c_double(stepsize)
)
# Convert results back to a NumPy array
results = np.ctypeslib.as_array(results_ptr, shape=(no_datapoints,))
# Generate x-values for plotting
x_values = np.arange(lowerbound + stepsize, upperbound + stepsize, stepsize)
# Calculate the y-values for the function y = -1/2 * x^2
y_function = -0.5 * x_values**2
# Plot the data
plt.scatter(x_values, results, color='blue', s=1, label='Sim.')
plt.plot(x_values, y_function, color='red', label='Theory')
plt.xlabel('x')
plt.ylabel('y')
plt.legend()
plt.grid(True)
plt.savefig('../figs/fig.png')
plt.show()
\end{lstlisting}
\end{frame}
\end{document}

